\documentclass[12pt]{article}
\usepackage{graphicx}
\usepackage{fancyhdr}
\usepackage{geometry}
\usepackage{xcolor}
\usepackage{pgfgantt}
\usepackage{url}
\geometry{margin=1in}

\title{Design Doc}
\author{Group 8}
\begin{document}
\pagestyle{fancy}
\section{Design Approach}
\subsection{Contactless Capacitive Water Level Sensing}
The strip will be placed vertically on the exterior side of the brita/pitcher to detect exact water level. 
Works as long as plastic is not too thick which britas fit the description. 

\subsubsection{Pros}
    \begin{itemize}
    \item Low-cost
    \item No contact
    \item Small mechanical approach
    \item Customizable for future addons
    \end{itemize}
\subsubsection{Cons}
    \begin{itemize}
    \item Need a good environment for sensor, not humid so not great inside a fridge
    \end{itemize}

\subsection{Weight based, Load}
The design places the pitcher unto a load cell to measure the weight. We will calibrate it to the changes in weight.
\subsubsection{Pros}
    \begin{itemize}
    \item Highest accuracy of all designs
    \item Fits to many pitcher dimensions
    \end{itemize}
\subsubsection{Cons}
    \begin{itemize}
    \item Need a good environment for sensor, not humid so not great inside a fridge
    \end{itemize}

\subsection{Ultrasonic}
An ultrasonic sensor estimates the water level by measuring the echo time of sound waves reflected from the water surface.
\subsubsection{Pros}
    \begin{itemize}
    \item Inexpensive
    \item Easy implementation
    \end{itemize}
\subsubsection{Cons}
    \begin{itemize}
    \item Humidity problem
    \end{itemize}

\subsection{Laser}
A laser or time-of-flight sensor measures the distance from the sensor to the water surface.
The measured distance is converted into liquid height and volume.
\subsubsection{Pros}
    \begin{itemize}
    \item Continous measurements
    \item Non contact
    \end{itemize}
\subsubsection{Cons}
    \begin{itemize}
    \item Humidity problem
    \item Clear line of sight
    \end{itemize}

    \section{Decision Matrix}

Criteria will be scored on a scale from 1 (poor) to 5 (excellent). Weights were assigned according to relative importance. The final score for is calculated as:

\[
\textbf{Total Score} = sum(weight × score)
\]

\subsection{Evaluation Criteria}

\begin{itemize}
\item \textbf{Cost}: Estimated hardware and implementation expense
\item \textbf{Power}: Power consumption and low-power operation
\item \textbf{Complexity}: Mechanical, electrical, software integration difficulty
\item \textbf{Measurement Data}: Quality of information provided (Continous water level, binary off or on, ability to track water usage)
\end{itemize}

\subsection{Weighted Decision Matrix}

\begin{table}[h]
\centering
\begin{tabular}{lcccc}
\hline
\textbf{Criteria (Weight)} & \textbf{Capacitive} & \textbf{Weight} & \textbf{Laser} & \textbf{Ultrasonic} \\
\hline
Cost (0.30) & 5 & 4 & 3 & 5 \\
Power (0.20) & 5 & 4 & 3 & 4 \\
Complexity (0.25) & 5 & 3 & 2 & 3 \\
Measurement Data (0.25) & 3 & 5 & 3 & 3 \\
\hline
\textbf{Weighted Total} & \textbf{4.40} & \textbf{3.95} & \textbf{2.70} & \textbf{3.85} \\
\hline
\end{tabular}
\caption{Decision matrix comparing feasible sensing approaches (1 = poor, 5 = excellent)}
\end{table}




\end{document}