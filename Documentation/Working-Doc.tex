\documentclass[12pt]{article}
\usepackage{graphicx}
\usepackage{fancyhdr}
\usepackage{geometry}
\usepackage{xcolor}
\usepackage{pgfgantt}
\usepackage{url}
\geometry{margin=1in}

\title{CSE 123A - Winter 2026}
\author{Group 8}
\date{}

\begin{document}
\pagestyle{fancy}
\maketitle

\fancyfoot{}
\fancyhead{}
\fancyfoot[R]{\thepage}

% Problem Statement
\section*{Problem statement}
Living with roommates can sometimes mean your filtered water device is empty when you need water. This can be very frustrating and leave you needing to drink tap water while other roommates get to drink cold refrigerated and filtered water. 

% Need Statement
\section*{Need Statement}
\textbf{We want a way to know if the container is out of water when we need water. }
\begin{itemize}
    \item Avoid having no water left in addition to everybody using the water filter, not knowing there is no water left.

    \item Need a way to know the current water level of the pitcher so that it isn't empty without everyone knowing it's empty. 
    
    \item We need to know if the container is empty, in order to not run out of clean fresh water. 
\end{itemize}


% Goal
\section*{Goal}
\begin{itemize}
    \item Keeping water level known at all times, while informing to everybody in the household if the water level runs too low.
    \item Create a system to inform users if the container is empty. 
\end{itemize}
\textbf{If the container is empty, inform users.}

% Personas and Users
\section*{Personas and Users}
\textbf{This is intended for shared households that rely on a shared filtration water container}
\begin{itemize}

    \item College Roommates
    \begin{itemize}
        \item John Doe, 20 years old, Student at UCSC. John lives in a dorm with two other roommates, and they all rely on a shared water filter for drinking water. Occasionally, John comes out to get water only to find that the water is empty. This creates frustration and delays as John has to choose between missing the bus to refill water or get to class on time while being thirsty.
    \end{itemize}
    
    \item Family
    \begin{itemize}
        \item Jane Smith, 39 years old, working mother. Jane lives with her husband and two children. Jane gets home after a long day of work and is feeling thirsty from the drive home. She goes to the shared water filter to find that it is completely empty. Jane feels frustrated with her stay-at-home husband and two children for not refilling the water filter. This causes her to lash out and yell at them until they all cry, creating a hostile environment in the household.
    \end{itemize}
    
    \item Office
    \begin{itemize}
        \item Joe Micheal, 30 years old, works with Excel spreadsheets. He has very limited time to refill his mug, and his boss, George, never refills the water filter. As a result, Joe's efficiency has substantially decreased at his desk.
    \end{itemize}
    
\end{itemize}

% TimeLine3
\section*{Timeline}

\noindent\resizebox{\textwidth}{!}{%
\begin{ganttchart}[
    hgrid,
    vgrid,
    x unit=1.2cm,
    y unit chart=0.6cm,
    bar height=0.5,
    bar/.append style={fill=gray!35},
    title/.append style={fill=gray!15},
    title label font=\bfseries
]{4}{10}
\gantttitle{Weeks 4--10}{7} \\
\gantttitlelist{4,...,10}{1} \\

\ganttbar{Goal 1}{4}{4} \\
\ganttbar{Goal 2}{4}{5} \\
\ganttbar{Goal 3}{5}{6} \\
\ganttbar{Goal 4}{6}{7} \\
\ganttbar{Goal 5}{7}{8} \\
\ganttbar{Goal 6}{8}{9} \\
\ganttbar{Goal 7}{9}{10} \\
\ganttbar{Goal 8}{10}{10}

\end{ganttchart}%
}

\end{document}
