% mention collabrative tools that we use
% format
% - waht we did 
% what we are currently working
% what to do in the future



%todo:
% script for demonstration
% timeline

% status report 3

% put gitlab link in report and give access to prof and ta
% https://git.ucsc.edu/pnauwela/123a-project/-/tree/d3ba4457ead67f81f786f53da5bf10631144b1ea/

\documentclass[12pt]{article}
\usepackage{graphicx}
\usepackage{fancyhdr}
\usepackage{geometry}
\usepackage{hyperref}
\usepackage{xcolor}
\usepackage{url}
\geometry{margin=1in}

\title{CSE 123A - Week 3 Status}
\author{Group 8}
\date{}

\begin{document}
\pagestyle{fancy}
\maketitle

\fancyfoot{}
\fancyhead{}
\fancyfoot[R]{\thepage}

% talk about what we did week
\section*{What we Did}
For this week we continued researching parts and other components we will need for our project. We also started to think about everything will be implemented and what we want to for our working demo. 
We also now have a few design and concept drawings demonstrating the first design we will be implementing. These drawings will likely change and become more detailed as we iterate over ideas. 

For parts, we have ordered the load cells and amplifier from Amazon and are waiting for them to arrive so we can start work on soldering. Code decisions have been made and we are in the process of setting up initial prototypes for each segment. 

\section*{In progress}
We just emailed BELS about the parts for that we will need for our design, and until they respond, we are researching what is the best type of 
load cell we should use. for the other components we will be using an ESP32c3 as the microcontroller, and we plan on hosting a server using Vercel and Supabase 
for the web app database. For the housing of the project, we will plan various designs and out of those we will 3D print each design and test them to see which one would be the best.

\section*{What we need to}
While waiting on all the components to arrive, we are working on the implementation of the web app and the server, and we are working on the design of the housing. 
We will also start building the firmware for the microcontroller and the sensor, so we can start testing and soon it arrives. For our prototype demonstration, we will have load cell sensor 
connected to our microcontroller to which all of it will be connected to a computer that will output the water level by showing either a graph, or a number that will indicate the water level. 
We will also have a web app that will show the water level, however it will not be fully connected to the microcontroller, but it will be able to show the water level by using a mock data that we will generate.

%Script for the demonstration



\end{document}