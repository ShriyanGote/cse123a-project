% status report 4

\documentclass[12pt]{article}
\usepackage{graphicx}
\usepackage{fancyhdr}
\usepackage{geometry}
\usepackage{hyperref}
\usepackage{xcolor}
\usepackage{url}
\geometry{margin=1in}

\title{CSE 123A - Week 4 Status}
\author{Group 8}
\date{}

\begin{document}
\pagestyle{fancy}
\maketitle

\fancyfoot{}
\fancyhead{}
\fancyfoot[R]{\thepage}

 % whats been completed 
\section*{What we Did}

% talk about the sensor we are setting up and designing a base for

We have purchased the power module for this device. For the prototype we will be using a 3.7V 950mAh lithium ion battery. It will be connected to the ESP32 using the battery contact pins and a soldered on connector. Once we have set the device up in a way that we don't need to talk to it using USB for testing, we can power it solely from this battery. It should have more than enough power to support the board and the connected sensor. 
We also began designing custom 3D-printed mounting plates for the Digital Load Cell Weight Sensor. The circular plates (110mm diameter) went through several design iterations to get the M4 screw hole positioning and sizing right, with test prints done to verify fitment against the actual load cell.
 
% whats in progress
\section*{What we are Working On}
We are currently calibrating the load cell sensor now that the mounting plates have been built and tested. This involves configuring the HX711 breakout board and tuning the readings to ensure accurate weight measurements. We are also working on an improved plate design that includes countersunk screw holes so the base sits flat on a surface, as the current version leans slightly due to the screw heads protruding from the bottom.

% what next on teh agenda 
\section*{What's next}

% also have indivudal secitons of what each persion did and is working on

\section*{Individual Contributions}

\noindent\textbf{Dev Chodavadiya}:
\begin{itemize}
    \item Designed custom circular mounting plates (110mm diameter) for the Load Cell Weight Sensor using OpenCAD
    \item Went through multiple design revisions to achieve proper M4 screw hole placement, aligning holes at 22mm and 32mm from the 
        plate center to match the load cell mounting points
    \item Test printed plates to verify fitment and adjusted hole diameter sizing to ensure proper M4 screw compatibility
\end{itemize}

\noindent Next steps include designing and printing spacers to go between the load cell and the mounting plates, as well as adding 
a recessed countersink around the screw holes on the bottom of the plate so the screw heads sit flush and the base lays flat on a surface.\\

\noindent\textbf{Logan Bossuwe}:
Worked on the initial design to begin calibrating. Took 3D printed plates the the floor and top plate for the load cell. Soldered header pins to the
HX711
\noindent\textbf{Pieter Nauwelaerts}:

\noindent\textbf{Reid Graham}:

\noindent\textbf{Sam Lai}:

\begin{itemize}
    \item Cloned the separate web server repository and tested if I can deploy to Vercel successfully
    \item Tested the server locally and made myself familiar with the codebase
    \item Researched ways to implement notifications to users when the water level is low
    \item Found two methods: text message notifications and push notifications.
\end{itemize}

\noindent I am leaning towards push notifications since they are more user friendly and can be implemented using a service like 
Firebase Cloud Messaging (FCM). Text message notifications would require integrating with a service like Twilio, which 
have additional costs, while FCM has a free tier that should be sufficient for our needs. I will be working on integrating 
FCM into our web server and setting up the necessary infrastructure to send push notifications to users when the water level 
is low. I will also be working on writing some unit tests for receiving sample water level data and triggering notifications.\\

\noindent\textbf{Shriyan Gote}:

\end{document}